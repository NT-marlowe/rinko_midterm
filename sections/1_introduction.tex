\section{Introduction}
In the contemporary digital landscape, the sophistication of malware,
particularly in its ability to evade detection and analysis,
poses a significant challenge to cybersecurity efforts.
Among these evasive tactics, shadow attacks~\cite{Weiqin:ShadowAttack},
which cleverly distribute malicious activities across multiple processes,
stand out as particularly insidious.
These attacks exploit the inherent complexity of operating systems,
mimicking benign multi-process behavior to obfuscate their malicious intent.
Traditional detection mechanisms, reliant on static and dynamic analysis techniques,
often fall short in identifying these distributed threats,
necessitating the exploration of more advanced methodologies.

This paper introduces an innovative approach to tackling the
challenge posed by shadow attacks through the use of Extended Berkeley Packet Filter (eBPF).
eBPF, a technology that allows for the safe execution of custom code within the
Linux kernel without changing kernel source code or loading kernel modules,
offers a powerful mechanism for monitoring and tracing system-level operations.
Our research leverages eBPF to analyze the interconnections between function calls,
thereby revealing the execution patterns of processes involved in shadow attacks.
By mapping these patterns, we aim to uncover the stealthy operations of evasive malware,
providing insights that could lead to more effective detection and mitigation strategies.

We focus on the potential of eBPF to provide granular visibility into the behavior
of systems at runtime, enabling the identification of the complex orchestration of
processes characteristic of shadow attacks. Through the detailed analysis of function
call chains, we can trace the flow of execution within malicious processes,
identifying their strategies and mechanisms. This approach not only enhances
our understanding of how such attacks are constructed and executed but also opens
new avenues for developing countermeasures that can detect and neutralize these threats more efficiently.

By employing eBPF to dissect the intricacies of process execution and interaction in the
context of shadow attacks, we believe our research will contribute a novel perspective to the field of
cybersecurity. We demonstrate how eBPF's capabilities can be harnessed to advance our
understanding of malicious process execution, offering a promising methodology for combatting
evasive malware. Through this work, we aim to bolster the cybersecurity community's arsenal against
the ever-evolving landscape of malware threats, ensuring a more secure digital environment for all users.

Our exploration into the use of eBPF against shadow attacks not only highlights the adaptability and
complexity of modern malware but also underscores the necessity for innovative detection and analysis
techniques. As we delve into the capabilities and applications of eBPF, we pave the way for future
research and development in the domain of cybersecurity, seeking to establish more sophisticated defenses
against the cunning and elusive nature of malware attacks.
