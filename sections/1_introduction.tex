\section{Introduction}
In the contemporary digital landscape, the sophistication of malware,
particularly in its ability to evade detection and analysis,
poses a significant challenge to cybersecurity efforts.
Among these evasive tactics, shadow attacks~\cite{Weiqin:ShadowAttack},
which cleverly distribute malicious activities across multiple processes,
stand out as particularly insidious.
These attacks exploit the inherent complexity of operating systems,
mimicking benign multi-process behavior to obfuscate their malicious intent.
Traditional detection mechanisms, reliant on static and dynamic analysis techniques,
often fall short in identifying these distributed threats,
necessitating the exploration of more advanced methodologies.

In the contemporary digital realm, the sophistication of malware in its evasion of detection and analysis presents a considerable obstacle to cybersecurity endeavors. Among the various strategies employed, shadow attacks\cite{Weiqin:ShadowAttack},
are noteworthy for their adept distribution of malicious actions across multiple processes.
Exploiting the intricate nature of operating systems, these attacks mimic benign behavior
involving multiple processes to conceal their malevolent intentions.
Conventional detection methods, which rely on both static and dynamic analysis approaches,
often prove inadequate in detecting these dispersed threats, thus necessitating the exploration of more sophisticated methodologies.

This paper presents a novel strategy for addressing the challenge presented by shadow assaults using the Extended Berkeley Packet Filter (eBPF). eBPF, a technology enabling the secure implementation of customized code within the Linux kernel without altering kernel source code or loading kernel modules, provides a robust mechanism for monitoring and tracing system-level activities. Our study utilizes eBPF to scrutinize the relationships between function invocations, thus unveiling the operational sequences of processes implicated in shadow assaults. Through the depiction of these sequences, our objective is to expose the covert activities of evasive malicious software, yielding perspectives that might facilitate the development of more efficient detection and alleviation approaches.

We focus on the potential of eBPF to provide granular visibility into the behavior
of systems at runtime, enabling the identification of the complex orchestration of
processes characteristic of shadow attacks. Through the detailed analysis of function
call chains, we can trace the flow of execution within malicious processes,
identifying their strategies and mechanisms. This approach not only enhances
our understanding of how such attacks are constructed and executed but also opens
new avenues for developing countermeasures that can detect and neutralize these threats more efficiently.

By utilizing eBPF for the analysis of the complexities involved in process execution and interactions within the realm of shadow attacks, it is posited that our study will present a new viewpoint to the domain of cybersecurity. The utilization of eBPF's functionalities is showcased as a means to enhance our comprehension of malevolent process execution, providing a potentially effective approach to combating elusive malware. The objective of this research endeavor is to enhance the repertoire of tools available to the cybersecurity community in response to the perpetually changing landscape of malware risks, thereby ensuring a more resilient digital environment for all stakeholders.

Our investigation into the utilization of eBPF to combat shadow attacks not only emphasizes the flexibility and intricacy of contemporary malicious software but also stresses the importance of pioneering detection and examination methodologies. Delving into the functionalities and utilizations of eBPF lays the foundation for forthcoming exploration and enhancement in the field of cybersecurity, aiming to establish advanced defenses against the crafty and evasive characteristics of malware incursions.
