\section{Related Work}
\subsection{Malware Analysis}
Malware analysis is a critical aspect of cybersecurity, enabling the identification
and mitigation of malicious software threats.
Analysis techniques can be roughly categorized into two classes: static and dynamic.

\subsubsection{Static Analysis}
Static analysis involves examining the structure and content of a suspected malicious file without executing it.
Static analysis are often conducted as a preliminary step to grasp the general characteristics of the malware
before more sophisticated analysis techniques are applied \cite{chakkaravarthy2019survey}.

\textbf{Signature matching} is a common technique that compares the "signature", characteristics of a file
against a database of malware such as VirusTotal \cite{VirusTot28:online}. The characteristics include file size, hash values, and byte sequences.
Although this method has been a staple in malware detection, it is limited by its inability to detect only known malware.

\textbf{Disassemble and decompile} is another static analysis technique
that involves converting the binary code of a malware into a human-readable format.
From the decompiled code, analysts might be able to identify the malware's functionality and behavior as well as finding
interesting strings like URLs, IP addresses, and encryption keys.

\subsubsection{Dynamic Analysis \cite{10.1145/3329786}}
Dynamic analysis entails examining the behavior of malware by executing it in a controlled environment.
This approach enables analysts to witness the interactions between the malware and the system,
providing insights into its true intentions and capabilities.

\textbf{Function call analysis} is a method that focuses on tracking functions issued by the malware and the parameters
passed to them. One way to archive this is by code injection, in which analyzing code is hooked into
a specific function call and various information is collected and notified when the function is called.
Carsten et al. \cite{willems2007toward} created an automated malware analysis system that injects DLLs within CWSandbox,
letting analysts monitor system calls.

\textbf{Data flow tracking} is another approach that tracks the flow of data through the malware, and data tainting is an established
technique in this area. It involves marking (or "tainting") specific data components and then monitoring
how this tainted data propagates through the system.
Data tainting could be utilized in static analysis, but due to some evasion strategies like encryption and obfuscation
it has endured challenges in practice \cite{alashjee2019dynamic}.
SELECTIVETAINT \cite{chen2021selectivetaint} was invented to address performance overhead issue by employing static binary
rewriting to selectively instrument only instructions related to taint analysis.

\subsection{Kernel Modules}
Kernel modules are a mechanism that allows for the extension of kernel functions without modifying the kernel's
source code by loading object files during the execution of the Linux kernel.
Kernel modules are not subject to constraints like the eBPF verifier, thus offering a high degree of program freedom.
However, since kernel modules are executed with the same privileges as the kernel,
it is necessary to develop carefully to avoid embedding vulnerabilities \cite{chen2011linux}.

As Mayer et al. \cite{mayer2021performance} point out, avoiding the creation of kernel modules can be
considered an advantage of eBPF.
